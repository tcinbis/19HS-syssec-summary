% !TeX root = ../summary-syssec.tex

\section{Basics}
\subsection{Concepts}
\begin{tabular}{p{2.2cm}p{2.5cm}p{2.5cm}}
  \hline
	& 	\textbf{Confidentiality} & \textbf{Authentification}\\
	\hline
	\hline
  \textbf{Symmetric cryptography} & symmetric key encryption & symmetric
  key authentication\\
  \hline
  \textbf{Asymmetric cryptography} & public key encryption & digital signatures\\
  \hline
\end{tabular}
\subsection{RSA}
\begin{enumerate}
	\item Choose primes $p,q$ and let $n = p\cdot q$
	\item Let  $\phi(n) = (p-1)(q-1)$
	\item Choose $e$ relatively prime to  $\phi(n)$: gcd($\phi(n)$, $r$) = 1
      \item Compute $d$, s.t. $ed \text{ mod } \phi(n) = 1$
	\item The public key is: PU = ($e$, $n$)
	\item The private key is: PK =  $(d, n)$
	\item Encrypt with $c = m^e \text{ mod } n $
	\item Decrypt with $m = c^d \text{ mod } n$
\end{enumerate}
Intuition: If one can factor $n$, one can break RSA.

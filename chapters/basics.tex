% !TeX root = ../summary-syssec.tex

\section{Basics}
\subsection{Concepts}
\begin{description}
  \item[Integrity:] The data was not changed by a party unauthorized to change
    it.
  \item[Confidentiality:] An unauthorized person cannot understand the contents of the
    message, because the message looks like random bits (protected data belongs to
    some other users).
  \item[Secrecy:] Same as confidentiality, but protected data belongs to the
    sender.
  \item[DoS:] A service is blocked either by sending enough traffic to overwhelm the service
    or selectively exploit bugs/errors in the design.
  \item[Authentication:] The identity of the sender is correct.
  \item[Authorization:] Examine whether a person has the capability to use
    a certain service or access certain data.
\end{description}

\begin{tabular}{p{2.2cm}p{2.5cm}p{2.5cm}}
  \hline
	& 	\textbf{Confidentiality} & \textbf{Authentification}\\
	\hline
	\hline
  \textbf{Symmetric cryptography} & symmetric key encryption & symmetric
  key authentication\\
  \hline
  \textbf{Asymmetric cryptography} & public key encryption & digital signatures\\
  \hline
\end{tabular}
\subsubsection{Proper Usage}
\begin{description}
  \item[Authenticate-then-encrypt:] provides both secrecy and authenticity of the message.
    \begin{itemize}
      \item susceptible to chosen ciphertext attacks, because an attacker can simply send messages and observe the system’s behavior upon
	decryption attempts.
      \item can potentially be used for a DoS attack
    \end{itemize}
  \item[Ecrypt-then-authenticate:] provides both secrecy and authenticity of the message.
    \begin{itemize}
      \item crafted messaged are detected immediately
      \item faster
    \end{itemize}
  \item[Encrypt-and-authenticate]
    The MAC output may reveal information on the plaintext and thus insecure.
\end{description}
\subsection{RSA}
\begin{enumerate}
  \item Choose primes $p,q$ and let $n = p\cdot q$
  \item Let  $\phi(n) = (p-1)(q-1)$
  \item Choose $e$ relatively prime to  $\phi(n)$: gcd($\phi(n)$, $e$) = 1
  \item Compute $d$, s.t. $ed \text{ mod } \phi(n) = 1$
  \item The public key is: PU = ($e$, $n$)
  \item The private key is: PK =  $(d, n)$
  \item Encrypt with $c = m^e \text{ mod } n $
  \item Decrypt with $m = c^d \text{ mod } n$
\end{enumerate}
Intuition: If one can factor $n$, one can break RSA.

\subsection{Block Cipher}
Secret key = (map cleartext $\Leftrightarrow$ ciphertext)
\begin{itemize}
	\item Modern block ciphers use a key of $K$ bits to specify a random
subset of $2^K$ mappings.
\item If the selection of the $2^K$ mappings is random, the resulting
cipher will be a good approximation of the ideal block cipher.
\item \textbf{Diffusion:} If plaintext bit $i$ is changed, the ciphertext bit $j$ should
  change with $p=0.5$
\item \textbf{Confusion:} Each bit of $c$ should depend on the whole key
\end{itemize}

\subsection{Hash function}
Properties of a hash function: $y = H(x)$
\begin{description}
    \item[one-way] can not be reversed 
    \item[weak collision resistance] given x, can not find $x' \not= x \text{ s.t. } H(x) = H(x')$
    \item[strong collision resistance] can not find $x' \not= x \text{ s.t. } H(x) = H(x')$
\end{description}


% !TeX root = ../summary-syssec.tex

\section{Unix Access Control}
\subsection{Setuid}
When an executable file's \texttt{setuid} permission is set, users may execute that
program with a level of access that matches the user who owns the file.

The kernel ignores \texttt{setuid} on scripts for security reasons.
\subsection{Directory Permissions}
\begin{description}
  \item[read bit (\texttt{r})] allows to list the files within the directory
  \item[write bit (\texttt{w})] allows to create, rename, or delete files
    within the directory, and modify the directory's attributes
  \item[execute bit (\texttt{x})] allows  to enter the directory, and access
    files and directories inside
  \item[sticky bit (\texttt{T}, or \texttt{t} if the execute bit is set for
    \texttt{others})] states that files and directories within that directory
    may only be deleted or renamed by their owner (or root)
\end{description}

\subsection{\texttt{sudo}, \texttt{su}}

\texttt{sudo}:
\begin{itemize}
  \item used to gain root access using the user's own password.
\end{itemize}
\texttt{su}:
\begin{itemize}
  \item used to switch between users using the target user's password
\end{itemize}
\texttt{sudo} allows more fine-grained configuration and thereby more selective
privilege escalation. $\Rightarrow$ prefer \texttt{sudo} over \texttt{su}.


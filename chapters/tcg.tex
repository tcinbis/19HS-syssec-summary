% !TeX root = ../summary-syssec.tex

\section{Trusted Computing \& Attestation}
Acronyms:
\begin{description}
    \item[TCG] Trusted Computing Group
    \item[TPM] Trusted Platform Module
    \item[PCR] Platform Configuration Register
    \item[TCB] Trusted Computing Base
    \item[SRTM/DRTM] Static / Dynamic Root of Trust for Measurement
    \item[SLV] Secure Loader Block
    \item[TXT] Trusted Execution Technology
    \item[AIK] Attestation Identity Keys (for signing PCRs)
    \item[SRK] Storage Root Key
    \item[IEE] Isolated Execution Environment
    \item[DMA] Direct Memory Access
\end{description}
\subsection{TPM}
\subsubsection{Overview}
\begin{description}
    \item[Attestation] enables verifier to verify what SW is executing on untrusted device
\end{description}
Adversary model of TPM:
\begin{itemize}
    \item remote attacks 
        \begin{itemize}
            \item compromise OS and apps running on the OS
            \item complete control over network communication
        \end{itemize}
    \item local hardware
        \begin{itemize}
            \item generally trusted (because HW attack detection is hard)
            \item attacker can reboot, install malicious SW, attack malicious USB devices
        \end{itemize}
\end{itemize}

Basic TPM functions
\begin{itemize}
    \item PCR registers for integrity measurement chain $PCR_{new} = \texttt{SHA-1}(PCR_{old}|| \texttt{SHA-1}(data))$
    \item on-chip storage for SRK
    \item manufacturer certificate
    \item remote attestation with PCRs and AIK
    \item sealed storage with PCRs and SRK (only accessible under certain integrity measurement)
    \item random number generator
\end{itemize}
TPM is \textbf{passive} component and \textbf{not} tamper proof.

HW attacks possible because TPM is on LPC bus.

\subsubsection{Attested Boot a.k.a. Static Root of Trust}
Measure all executed SW to verify platform configuration. To verify compare measurements to known hashes of trusted SW.

Shortcomings:

\begin{itemize}
    \item TOCTTOU: measurement at load-time and not at runtime (inefficient against dynamic attacks)
    \item coarse-grained: TCP is entire system
    \item no guarantee of execution
    \item every system is different
\end{itemize}

\subsubsection{Dynamic Root of Trust a.k.a. Late Launch}
Idea: special CPU instruction to create IEE $\xrightarrow{}$ high assurance of code execution and remote attestation

Required computing primitives:
\begin{itemize}
    \item create IEE with \texttt{SKINIT/SENTER}
        \begin{enumerate}
            \item CPU softreset
            \item reset dynamic PCRs
            \item enable DMA protection
            \item send SLB to TPM
            \item execute SLB
        \end{enumerate}
    \item remote attestation
        \begin{enumerate}
            \item verifier sends nonce to TPM
            \item TPM returns measurements, nonce signed with platform key
        \end{enumerate}
    \item establish secure channel
        \begin{enumerate}
            \item verifier sends nonce to TPM
            \item TPM generates $\{K, K^{-1}\}$, send back $\{\texttt{measurement, nonce}, K\}$ signed with platform key
        \end{enumerate}
    \item verify output
        \begin{enumerate}
            \item verifier sends nonce to TPM
            \item TPM sends back measurement, nonce, input and output signed with platform key
        \end{enumerate}
\end{itemize}

No need to trust OS anymore, therefor minimal TCB.

\subsection{SGX vs TPM}
SGX advantages:
\begin{itemize}
    \item Memory encryption
    \item robust against LPC-bus tampering
    \item can execute unprivileged code (ring 3)
    \item multi-threaded execution
    \item parallel execution of enclave and untrusted code
    \item enclaves can be interrupted
\end{itemize}

SGX disadvantages:
\begin{itemize}
    \item no sealed storage
    \item remote attestation requires online third party (Intel)
    \item memory access pattern reveals information about computation
\end{itemize}

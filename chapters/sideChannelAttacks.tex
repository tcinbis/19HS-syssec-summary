% !TeX root = ../summary-syssec.tex

\section{Side Channel Attacks}
\subsection{Introduction}
Security proofs are based on models of system and attacker. Most models do not take
into account the implementation and how it interacts with the environment.

Examples:
\begin{itemize}
  \item Faulty Output (Glitch)
  \item Power Consumption
  \item EM Emissions
  \item Heat
  \item Timing
  \item Design Details
  \item Sound
  \item Data Coupling
\end{itemize}

\textbf{Two categories of Side channel analysis:}
\begin{description}
  \item[Simple] computation only depends upon the key.
  \item[Differential] computation depends upon both the input and the key.
\end{description}

\subsection{Timing Cryptanalysis of RSA}
\begin{enumerate}
  \item Execution time depends on the key
  \item Can be measured for different inputs
\end{enumerate}


\subsubsection{Square-And-Multiply}
\begin{lstlisting}
// Computes C = m^d mod n
// MSB first
Square-And-Multiply(m, d, n):
  w = bitlength(d)
  x = m
  for i = 1 to w-1:
    x = x*x  mod n
    if (bit i of d) == 1:
      x = x*m mod n
  return x
\end{lstlisting}

\begin{itemize}
  \item Key-dependent Branching
  \item Attacker observes how many 1 are in the key
\end{itemize}
Greatly reduces the search space for the secret key, but still huge...

\subsubsection{Montgomery multiplication}
Multiplication the execution time depends on \textbf{input and key}.

Attack main idea:
\begin{itemize}
  \item Process attack bit per bit.
  \item Try large number of different messages and leverage average execution
    time.
  \item For each message, test whether the round bit was 1 or 0.
\end{itemize}

\subsubsection{Protecting}
\begin{itemize}
  \item Change implementation s.t. it is not time dependent.
  \item Generic protection is hard.
  \item Performance Penalty
\end{itemize}


\subsection{Cache Attacks}
\subsubsection{Flush+Reload}
For shared memory only.
\begin{enumerate}
	\item Attacker flushes cache
	\item Victim caches data
	\item Attacker accesses possible cached data. If this access was fast,
	  then data is cached.
\end{enumerate}
\subsubsection{Prime+Probe}
    Also without shared memory (\textbf{NOTE:} we only get the cache address
    which was loaded by victim)

\includegraphics[width=\columnwidth]{sidechannel-prime-probe.png}


\subsubsection{Variants}
\begin{enumerate}
  \item \textbf{Data access} depends on secret data: Data access patterns
    in the cache leak information about the secret
  \item \textbf{Control flow} depends on secret data: Code access
    patterns in the cache leak information about the secret
\end{enumerate}
\subsubsection{Cache Timing Attack on AES}
\textbf{Background}\\
During the execution of AES secret key is used to
index arrays (S-boxes). The time to lookup an array element depends
on whether this element of the S-box has partially or entirely been loaded into the
cache.
\begin{itemize}
  \item Allows complete key recovery.
  \item Problem is in the design, not in the implementation.
\end{itemize}

\textbf{Setting:}
\begin{itemize}
  \item Attacker can send messages and observe overall execution time
  \item Attacker has same setup (software, hardware) for local tests
    Execution
\end{itemize}

\textbf{Attack:}
\begin{itemize}
  \item For each byte of the key, one index value will have the slowest lookup.
  \item The total execution will be slow
  \item Find this index value
\end{itemize}
$\Rightarrow$  try many messages locally


\subsubsection{Protecting}
\begin{itemize}
  \item \textbf{Basic principle:}
    Try to eliminate secret-dependent cache access patterns
  \item Specific defenses exist, but eliminating all secret-dependent
    branching (code and data) is difficult and expensive…
  \item Most CPUs have special AES hardware  $\Rightarrow$ not vulnerable.
\end{itemize}

\subsection{Power Analysis Attacks}
Mainly used on Smartcards, RFID chips, Sensor Nodes
\begin{itemize}
  \item The attacker needs to have physical access
  \item Measures the consumed power during  operation
\end{itemize}
Square-and-multy in RSA is vulnerable through simple power analysis: Attacker
can get entire key.
\subsubsection{Protection}
Goal: Elimination or significant reduction of the correlation between operand
values and power consumption.
\begin{itemize}
  \item Random change of power consumption in time
  \item Noise generator
  \item Physical shielding
  \item Software balancing
  \item Hardware balancing
\end{itemize}

\subsection{Acoustic Attacks}
\begin{itemize}
  \item High-frequency sounds caused by vibration of
    electronic components
  \item \textbf{Different keys cause different sounds}
  \item Can extract RSA keys based on sound
\end{itemize}

\subsection{Electromagnetic Attacks}
TEMPEST: Transmitted Electro-Magnetic Pulse / Energy
Standards \& Testing
\begin{itemize}
  \item Compromising emanations may be generated by any electrical information generating or processing equipment.
  \item Can be sensed and transmitted over air, water, electrical lines, ...
  \item Typical examples include:
    \begin{itemize}
      \item Displays
      \item Keyboards
      \item Cables
      \item Processors
    \end{itemize}
\end{itemize}

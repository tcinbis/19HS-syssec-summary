% !TeX root = ../summary-syssec.tex

\section{Side Channel Attacks}
\subsection{Introduction}
Security proofs based on models of system and attacker. Most models do not take
into account the implementation and how it interacts with the environment.

Examples:
\begin{itemize}
  \item Faulty Output (Glitch)
  \item Power Consumption
  \item EM Emissions
  \item Heat
  \item Timing
  \item Design Details
  \item Sound
  \item Data Coupling
\end{itemize}

\subsection{Timing Cryptanalysis of RSA}
\begin{enumerate}
  \item its execution time depends on the key
  \item  it can be measured for different inputs
\end{enumerate}

\subsubsection{Square-And-Multiply}

\begin{itemize}
  \item Key-dependent Branching
  \item Attacker observes how many 1 are in the key
\end{itemize}
Greatly reduces the search space for the secret key, but still huge...

\subsubsection{Montgomery multiplication}
Multiplication the execution time depends on \textbf{input and key}.

Attack main idea:
\begin{itemize}
  \item Try large number of different messages and leverage average.
  \item Simulate execution up to round $i$ (using known key bits)
  \item Try many messages, measure execution time
\end{itemize}

\subsubsection{Protecting}
\begin{itemize}
  \item Change implementation s.t. it is not time dependent.
  \item Generic protection is hard.
  \item Performance Penalty
\end{itemize}

\subsection{Cache Attacks}


% !TeX root = ../summary-syssec.tex

\section{IoT}
\subsection{Light bulbs go nuclear}
\begin{enumerate}
    \item extract firmware update signing key using differential power analysis
    \item circumvent proximity checks
    \item make bulb join malicious network
    \item perform OTA firmware update
\end{enumerate}
Note: classical defence mechanisms like firewalls are not effective

Works because firmware only includes a MAC and is not signed (asymmetrically), therefor extracted key can \textit{sign} new firmware updates.

\subsubsection{IoT Stack}
\begin{center}
    \includegraphics[width=\linewidth]{images/iot_stack}
\end{center}

\subsubsection{Encrypt everything?}
\textbf{Not sufficient}! device identifiers, traffic volume still leak information, activities

\begin{description}
  \item[existential leakage] single transmission implies real-world event
    \begin{itemize}
      \item Can not be fully eliminated
    \end{itemize}
  \item[statistical leakage] deviation from normal implies real-world event
    \begin{itemize}
      \item Can be eliminated.
    \end{itemize}
\end{description}

\subsubsection{Hide transmission times and encrypt?}
\textbf{Not sufficient}! Examples:
\begin{itemize}
    \item VOIP word reconstruction from encrypted packages (leak from variable bit length encoding and length-preserving encryption).
    \item side-channel leaks of web apps by analysing traffic
\end{itemize}

\subsection{Device Pairing}
\begin{description}
    \item[Pairing] process of establishing a security association between two devices without prior shared knowledge
    \item[security association] can be anything only known to paired devices (e.g. secret key)
\end{description}

Paring method shall minimise I/O hardware needed and avoid complex user interactions.

\begin{description}
    \item[Out-of-band] extra channel which guarantees authenticity and is
      assumed to be secure even under MITM attack \hfill\\
      Examples:
      \begin{itemize}
      	\item Visible Light
	\item Infrared
	\item Audio
	\item Haptic
	\item Sensing
      \end{itemize}
    \item[In-band] often relies on physical channel properties and use special encoding (etc.) to detect MITM attack easily.
\end{description}

\subsubsection{Bluetooth Low Energy}
Four authentication methods supported:
\begin{description}
  \item[Just Works] provides no authentication.
  \item[Passkey Entry] displays a 6-digit PIN code on one device, and the user needs to
    type the PIN on the other.
  \item[Numeric Comparison] displays 6-digit PIN codes on both devices and the
    user needs to confirm that they are identical.
  \item[Out-of-band (OOB)] is directly provided with a shared key and assumes the key
    has been securely established via an alternative OOB channel.
\end{description}
Devices can negotiate and select one based on their I/O capabilities.

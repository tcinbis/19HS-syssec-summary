\section{OS Security}

\begin{description}
    \item[OS Security Goals] secrecy, integrity, availability
    \item[Trust Model] set of SW and data which the system depends on for correct enforcement of security goals
    \item[Threat Model] set of operations available to compromise system
\end{description}

\subsection{Access control}
Whether to allow requests from \textbf{subjects} to perform \textbf{operations} on \textbf{objects}.

\begin{description}
    \item[protection system] defines security requirements of OS. Consists of protection state and protection state operations
        \begin{description}
            \item[protection state] operations a subject can perform on obj.
            \item[protection state operations] enable modification of state
        \end{description}
\end{description}

An access matrix defines which operations a subject can perform on an obj. (subject as row, object as column)
Store by column: ACL (\textbf{A}ccess \textbf{C}ontrol \textbf{L}ist) stored with objects.
Store by row: Capability of a subject
\begin{center}
    \includegraphics[width=0.65\linewidth]{images/os_sec_capabilities-vs-ACLs.png}
\end{center}

\subsubsection{DAC}
\textbf{D}iscretionary \textbf{A}ccess \textbf{C}ontrol.
Is called discretionary, because subject can pass on his permissions directly/indirectly to other subjects.
\begin{itemize}
    \item allow untrusted processes/users to modify protection state
    \item only works if all processes are harmless and users make no mistakes
\end{itemize}
\subsubsection{MAC}
\textbf{M}andatory \textbf{A}ccess \textbf{Control}

\begin{itemize}
    \item only modifiable by admin
    \item subjects and obj are labels, state represents allowed operations
    \item transition state is legal way to relabel subjects and objs
\end{itemize}

\begin{center}
    \includegraphics[width=0.9\linewidth]{images/os_sec_MAC.png}
\end{center}

\subsubsection{Reference Monitor}
\begin{description}
    \item[input] request of security sensitive operation
    \item[output] binary answer
    \item[authorization module] converts request into query for policy store
\end{description}

\textbf{Secure OS} iff system uses reference monitor which satisfies:
\begin{enumerate}
    \item complete mediation
    \item tamperproof
    \item verifiable
\end{enumerate}

Does not protect against TOCTTOU attacks.

\subsubsection{Covert Channels}
Capability to transfer information between processes that are not supposed to be allowed to communicate by the security policy.

There are \textbf{storage} and \textbf{timing} channels.

Prevent it by:
\begin{itemize}
    \item memoryless program
    \item isolate program
    \item transitivity (if program A is confined and calls program B, then B must also be confined)
    \item masking (caller determines all inputs to all channels)
    \item enforcement (enforce that input into covert channel conforms with caller's [security] specification)
\end{itemize}

\subsection{Unix Access Control}
\subsubsection{Chmod}
Only the owner  of a file (or \texttt{root}) can change file permissions. Even if a user has write access via group permissions he is not able to change it.
\subsubsection{Setuid}
When an executable file's \texttt{setuid} permission is set, users may execute that
program with a level of access that matches the user who owns the file.

The kernel ignores \texttt{setuid} on scripts for security reasons.
\subsubsection{Directory Permissions}
\begin{description}
  \item[read bit (\texttt{r})(\texttt{4})] allows to list the files within the directory
  \item[write bit (\texttt{w})(\texttt{2})] allows to create, rename, or delete files
    within the directory, and modify the directory's attributes
  \item[execute bit (\texttt{x})(\texttt{1})] allows  to enter the directory, and access files and directories inside (without execute permission we can \textbf{not} list, write or create files in the directory)
  \item[sticky bit (\texttt{T}, or \texttt{t} if the execute bit is set for
    \texttt{others})] states that files and directories within that directory
    may only be deleted or renamed by their owner (or root)
  \item[setuid] no effects
  \item[setgid] causes any file created in a directory to inherit the
    directory's group
\end{description}

\subsubsection{\texttt{sudo}, \texttt{su}}

\texttt{sudo}:
\begin{itemize}
  \item used to gain root access using the user's own password.
  \item users with sudo access defined in \texttt{/etc/sudoers} or part of \texttt{sudo} group.
\end{itemize}
\texttt{su}:
\begin{itemize}
  \item used to switch between users using the target user's password
\end{itemize}
\texttt{sudo} allows more fine-grained configuration and thereby more selective
privilege escalation. $\Rightarrow$ prefer \texttt{sudo} over \texttt{su}.

\subsection{Linux Security Model}
Process in user space has access rights based on user ID (UID).

UID 0 is reserved for \texttt{root}.
\begin{description}
    \item[kernel process] can access everything
    \item[root process] can order kernel process to access everything
\end{description}

Kernel verifies permissions before system call! Compares UID and GID (Group ID) of subject and object.

\subsubsection{Password change}
\begin{itemize}
  \item \texttt{/bin/passwd} has suid bit set and thus runs as root.
  \item This is needed to change the passwords in \texttt{/etc/passwd}.
  \item \texttt{/bin/passwd} can access \textbf{any} system ressource.
  \item $\Rightarrow$ must trust \texttt{/bin/passwd}
\end{itemize}
Many programs can run as root and thus can modify the
shadow password file!


\subsection{Linux Security Module}
Linux Security Module (LSM) is a framework with the ideas:
\begin{itemize}
  \item Hook security functions and security data structures
  \item Allow registration and initialization of security modules (e.g.
    SELinux)
  \item Allow certain security attributes to be available to userspace services
    through \texttt{/proc}
  \item Limit performance overhead
\end{itemize}


\subsection{SELinux}
No default superuser exists, because SELinuxs default is to deny access. Everything must be specifically granted.

\subsubsection{RBAC: Role based access control}
\begin{itemize}
  \item rules specify roles user can use
  \item rules specify when and where a user can transition into another role
\end{itemize}

\subsubsection{MLS:  multi level security}
\begin{itemize}
  \item handling of classified data where \textit{no read up} or \textit{write down} is allowed
  \item enforce by file system labeling
\end{itemize}

\subsubsection{Modes}
SELinux has the following modes:

\begin{description}
  \item[Disabled:] loaded, but not running. Only DAC is enforced
  \item[Permissive:]  Policy is not enforced, but would-be denials are logged
  \item[Enforcing:] Policy is enforced, access is denied based on rules
\end{description}
\subsubsection{Type enforcement}

In SELinux every subject/object has a \textbf{security context} which consists of three labels:\vspace{-1.5mm}
\begin{description}
    \item[user label] on a subject: user privileges, on an object: objects owner
    \item[role label]  on a subject: subject's role, on an object: no meaning (always defaults to \texttt{object\char`_r})
    \item[domain label] set of subjects and objects which are allowed to interact with each other
\end{description}
This is called \textbf{type enforcement}! SELinux can make two types of decisions:\vspace{-1.5mm}
\begin{description}
    \item[access decision] Is subject allowed to do something within it's domain?
    \item[transition decision] Is subject allowed to do somthing in another domain (transition into this domain)?
\end{description}
Hard part is not to grant access to files, but allowing \textbf{safe domain transitions}.

\subsubsection{Allow rule}
four elements:
\begin{description}
  \item[Source type:] Domain type of a process attempting access
  \item[Target type:] Type of an object being accessed by the process
  \item[Object class:] Class of object that the specified access is permitted
  \item[Permission:] Kind of access that the source type is allowed
\end{description}

Example rule for passwd (access to shadow file missing):\\
\texttt{allow user\char`_t passwd\char`_exec\char`_t: file $\{$ getattr execute $\}$; \\ allow passwd\char`_t passwd\char`_exec\char`_t: file entrypoint; \\ allow user\char`_t passwd\char`_t: process transition;}

\subsubsection{System call}
\begin{center}
  \includegraphics[width= 0.7\columnwidth]{os_sec_system-call.png}
\end{center}

\subsection{Other OSs}
\subsubsection{Qubes}
Security by compartmentalization. 1 qube equals 1 VM.

\subsubsection{Microkernels}
Idea is to minimize TCB. Follows least priviledge principle and removes many parts from kernel space.
\begin{center}
    \includegraphics[width=0.8\linewidth]{images/os_sec_MicrokernelVSMonolithic.png}
\end{center}

\textbf{sel4} is a microkernel which focuses on efficency and high assurance. Uses formal verification of code to guarantee correctness w.r.t. specification.

% !TeX root = summary-syssec.tex

\usepackage[english]{babel}
\usepackage[landscape, margin=1cm]{geometry}
\usepackage[dvipsnames]{xcolor}
\usepackage{amscd, amsmath, amssymb, blindtext, empheq, enumitem, multicol, parskip, esint}
\usepackage{mathtools}
\usepackage{graphicx}
\usepackage{multirow}
\usepackage{grffile}
\usepackage{colonequals}
\usepackage{textcomp}
\usepackage{gensymb}
\usepackage{tikz}
\usepackage{bm}
\usepackage{trfsigns}
\usepackage{mathrsfs}
\usepackage{extarrows} % for arrows and equal-signs with text under and over them

% License
\usepackage[type={CC},modifier={by-sa},version={4.0}]{doclicense}

% MATH
\allowdisplaybreaks
% (subsection.Nummer) für referenzierte Gleichungen
\numberwithin{equation}{subsection}
\mathtoolsset{showonlyrefs} % (Nummer) nur für referenzierte Gleichungen
\DeclareMathOperator{\arcsinh}{arcsinh}
\DeclareMathOperator{\sinc}{sinc}
\DeclareMathOperator{\arccosh}{arccosh}
\DeclareMathOperator{\arctanh}{arctanh}
\DeclarePairedDelimiter{\ceil}{\lceil}{\rceil}

% Fonts
\usepackage{lmodern}
\usepackage[QX]{fontenc}
%define old fonts
\DeclareOldFontCommand{\bf}{\normalfont\bfseries}{\mathbf}
\DeclareOldFontCommand{\sf}{\normalfont\sffamily}{\mathsf}

% Programming
\usepackage{listings}
\usepackage{ulem}
\newcommand{\del}[1]{\strut\sout{#1}}
\lstset{
  basicstyle=\footnotesize,        % the size of the fonts that are used for the code
  breakatwhitespace=true,          % sets if automatic breaks should only happen at whitespace
  breaklines=true,                 % sets automatic line breaking
  commentstyle=\color{green},      % comment style
  escapeinside={\%*}{*)},          % if you want to add LaTeX within your code
  keepspaces=true,                 % keeps spaces in text, useful for keeping indentation of code (possibly needs columns=flexible)
  keywordstyle=\color{blue},       % keyword style
  language=C,                      % the language of the code
  numbers=left,                    % where to put the line-numbers; possible values are (none, left, right)
  numbersep=5pt,                   % how far the line-numbers are from the code
  numberstyle=\tiny\color{gray}, % the style that is used for the line-numbers
  xleftmargin=15pt,
  stepnumber=1,                    % the step between two line-numbers. If it's 1, each line will be numbered
  tabsize=2,	                   % sets default tabsize to 2 spaces
  moredelim    = [is][\del]{>->}{<-<},
}

% make document compact
\usepackage[compact]{titlesec}
\titlespacing{\section}{0pt}{*0}{*0}
\titlespacing{\subsection}{0pt}{*0}{*0}
\titlespacing{\subsubsection}{0pt}{*0}{*0}

\parindent 0pt
\pagestyle{empty}
\setlength{\unitlength}{1cm}
\setlist{leftmargin = *}

% define some colors
\definecolor{titletext}{RGB}{255,255,255}
\definecolor{formula}{RGB}{60,179,113}
\definecolor{subsection}{RGB}{77,209,77}
\definecolor{subsubsection}{RGB}{137,225,137}

% section color box
\setkomafont{section}{\mysection}
\newcommand{\mysection}[1]{%
    \Large\sf\bf%
    \setlength{\fboxsep}{0cm}%already boxed
    \colorbox{RoyalBlue}{%
        \begin{minipage}{\linewidth}%
            \vspace*{2pt}%Space before
            {\color{titletext} #1}
            \vspace*{-1pt}%Space after
        \end{minipage}%
    }}
%subsection color box
\setkomafont{subsection}{\mysubsection}
\newcommand{\mysubsection}[1]{%
    \normalsize \sf\bf%
    \setlength{\fboxsep}{0cm}%already boxed
    \colorbox{MidnightBlue}{%
        \begin{minipage}{\linewidth}%
            \vspace*{2pt}%Space before
             {\color{titletext} #1}
            \vspace*{-1pt}%Space after
        \end{minipage}%
    }}
%subsubsection color box
\setkomafont{subsubsection}{\mysubsubsection}
\newcommand{\mysubsubsection}[1]{%
    \normalsize \sf%
    \setlength{\fboxsep}{0cm}%already boxed
    \colorbox{Blue}{%
        \begin{minipage}{\linewidth}%
            \vspace*{2pt}%Space before
             {\color{titletext} #1}
            \vspace*{-1pt}%Space after
        \end{minipage}%
    }}

%MP
\newcommand{\MP}[2]{\begin{minipage}{#1\textwidth}#2\end{minipage}}
% equation box
\newcommand{\eqbox}[1]{\fcolorbox{black}{formula}{\hspace{0.5em}$\displaystyle#1$\hspace{0.5em}}}
% vectors in Math mode
\newcommand{\fat}[1]{\text{\textbf{#1}}}
% infty integral
\newcommand{\intf}[0]{\int_{-\infty}^{\infty}}
%inner product
\newcommand{\innerp}[2]{\langle#1,#2\rangle}
\newcommand{\innerpf}[2]{\langle\fat{#1},\fat{#2}\rangle}
